\item In un liceo viene svolta un'indagine sulle mete delle gite preferite dalle varie classi. I risultati sono stati: 	\newline
		\begin{center}
				\begin{tabular}{|c|c|}
				\hline
				Meta della gita & N classi \\
				\hline
				Firenze & 3 \\
				Perugia & 2 \\
				Roma  	& 5 \\
				Venezia	& 6 \\
				Francia	& 4 \\
				\hline
			\end{tabular}
		\end{center}
		Quali sono la popolazione statistica, l'unità statistica e il carattere (problema) coinvolti?\newline
		Disegna l'istogramma delle frequenze e determina la moda.