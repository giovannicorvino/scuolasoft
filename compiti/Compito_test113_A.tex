\documentclass[11pt,fleqn]{article} % use larger type; default would be 10pt

\usepackage[utf8]{inputenc} % set input encoding (not needed with XeLaTeX)
\usepackage{graphicx,amsmath} % support the \includegraphics command and options
\usepackage{booktabs} % for much better looking tables
\usepackage{array} % for better arrays (eg matrices) in maths
\usepackage{paralist} % very flexible & customisable lists (eg. enumerate/enumerate, etc.)
\usepackage{verbatim} % adds environment for commenting out blocks of text & for better verbatim
\usepackage{subfig} % make it possible to include more than one captioned figure/table in a single float

\setlength{\mathindent}{0pt}
\setlength{\topmargin}{-3.5cm}
\setlength{\oddsidemargin}{-0.5cm}
\setlength{\textwidth}{17cm}
\setlength{\textheight}{26cm}

\newcommand{\T}{\cdot}
\newcommand{\st}{\left(}
\newcommand{\dt}{\right)}
\newcommand{\sq}{\left[}
\newcommand{\dq}{\right]}
\newcommand{\sg}{\left\{}
\newcommand{\dg}{\right\}}
\newcommand{\f}{\frac}

\title{Test di ingresso - II D}
\date{2017-08-09}
\begin{document}
\maketitle
\begin{enumerate}

\item Metti in ordine crescente i seguenti numeri: 3,2 / 0,56 / 3 / 1,8 / 1,12 / 5 [1] 
\item Esegui in colonna le seguenti operazioni: 641+34 / 88-54 / 12$\T$45 / 364:28 [1] 
\item Scrivi l'espressione che risolve il seguente problema e risolvila: Il signor Gianni da un vivaista ha acquistato alcune piante da giardino per un totale di 65,80 euro e del terriccio a 13,40 euro. Se ha pagato con una banconota da 100 euro, quanto ha ricevuto di resto? [1] 
\end{enumerate}
\end{document}