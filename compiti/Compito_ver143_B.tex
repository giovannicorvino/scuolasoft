\documentclass[11pt,fleqn]{article} % use larger type; default would be 10pt

\usepackage[utf8]{inputenc} % set input encoding (not needed with XeLaTeX)
\usepackage{graphicx,amsmath} % support the \includegraphics command and options
\usepackage{booktabs} % for much better looking tables
\usepackage{array} % for better arrays (eg matrices) in maths
\usepackage{paralist} % very flexible & customisable lists (eg. enumerate/enumerate, etc.)
\usepackage{verbatim} % adds environment for commenting out blocks of text & for better verbatim
\usepackage{subfig} % make it possible to include more than one captioned figure/table in a single float

\setlength{\mathindent}{0pt}
\setlength{\topmargin}{-3.5cm}
\setlength{\oddsidemargin}{-0.5cm}
\setlength{\textwidth}{17cm}
\setlength{\textheight}{26cm}

\newcommand{\T}{\cdot}
\newcommand{\st}{\left(}
\newcommand{\dt}{\right)}
\newcommand{\sq}{\left[}
\newcommand{\dq}{\right]}
\newcommand{\sg}{\left\{}
\newcommand{\dg}{\right\}}
\newcommand{\f}{\frac}

\title{Verifica di aritmetica - I A - Compito B}
\date{2017-08-24}
\begin{document}
\maketitle
Punteggio per la sufficienza: 2
\begin{enumerate}

\item Esegui in colonna le seguenti operazioni: 897+798=.......... 123,7+7,03=.......... 1203-909=.......... 67,07-56,19=..........
\begin{figure}[h]
	\centering
		\includegraphics[width=13cm]{figure/quadretti.png}
\end{figure}
 [3] 
\item Esegui in colonna le seguenti operazioni: 246$\T$8=......... 12,1$\T$1,1=.......... 1701:3=.......... 12,5:0,5=..........
\begin{figure}[h]
	\centering
		\includegraphics[width=13cm]{figure/quadretti.png}
\end{figure} [3] 
\end{enumerate}
\end{document}