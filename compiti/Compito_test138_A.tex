\documentclass[11pt,fleqn]{article} % use larger type; default would be 10pt

\usepackage[utf8]{inputenc} % set input encoding (not needed with XeLaTeX)
\usepackage{graphicx,amsmath} % support the \includegraphics command and options
\usepackage{booktabs} % for much better looking tables
\usepackage{array} % for better arrays (eg matrices) in maths
\usepackage{paralist} % very flexible & customisable lists (eg. enumerate/enumerate, etc.)
\usepackage{verbatim} % adds environment for commenting out blocks of text & for better verbatim
\usepackage{subfig} % make it possible to include more than one captioned figure/table in a single float

\setlength{\mathindent}{0pt}
\setlength{\topmargin}{-3.5cm}
\setlength{\oddsidemargin}{-0.5cm}
\setlength{\textwidth}{17cm}
\setlength{\textheight}{26cm}

\newcommand{\T}{\cdot}
\newcommand{\st}{\left(}
\newcommand{\dt}{\right)}
\newcommand{\sq}{\left[}
\newcommand{\dq}{\right]}
\newcommand{\sg}{\left\{}
\newcommand{\dg}{\right\}}
\newcommand{\f}{\frac}

\title{Test di ingresso - II E}
\date{2017-08-24}
\begin{document}
\maketitle
\begin{enumerate}

\item Qual'è il più grande numero di 3 cifre con la cifra delle decine uguale a 5? [1] 
\item Nel seguente testo sottolinea i numeri ordinali e cerchia i cardinali: Giulia ha cinque gattini. Il più piccolo, Kitti, era l'ottavo di una cucciolata. Kitti ha l'abitudine di mettersi nei guai. Ieri si è infilato in un giardino in cui c'erano due enormi cani che l'hanno rincorso, fino a che non è riuscito a nascondersi sotto una vecchia 600. La prima volta che Kitti si è arrampicato su un albero, non è stato capace di scendere da solo, tanto che Giulia era tentata di chiamare il 115 sperando nell'aiuto dei vigili del fuoco. Non è stato però necessario, perché il papà di Giulia con una lunga scala a pioli ha tratto in salvo il micio birichino.  [1] 
\end{enumerate}
\end{document}